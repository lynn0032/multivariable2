\documentclass{ximera}

\graphicspath{{./graphics/}}

\title{Vector Fields}
\begin{document}
\begin{abstract}
\end{abstract}
\maketitle

In this activity, we introduced vector fields. We give examples and learn how to graph them.

\section*{Motivation}

Suppose we want to look at temperature patterns across a region at some specific time. We might represent this by taking the temperature at various locations, and then plotting each temperature at its location on a map.

PICTURE

Now, suppose we want to represent wind patterns across the same region at some specific time. We can represent this by plotting the wind speeds at various locations, but this doesn't tell the whole story - we'd also like to represent the direction in which the wind is blowing.

In order to do this, we can instead plot a \emph{vector} at each location. The direction of the vector will tell us the direction of the wind, and the length of the vector will tell us the wind speed.

PICTURE

Although we only plot the vectors at a few points, we know that there are windspeed vectors that could be drawn at every point in the region. However, we can use this plot to infer the behavior of windspeed in general, without checking the windspeed everywhere in the region.

This is an example of a vector field. There are many other important examples of vector fields, and they can be used to represent fluid flow, gravitational fields, and magnetic fields. 

PICTURES/EXAMPLES

\section*{Definition}

We've seen that a vector field consists of vectors placed at each point in some region, and we can think of this as assigning a vector to each point in the region. This sounds like a function, and provides the intuition behind our definition of a vector field.

\begin{definition}
A \emph{vector field} is a function $\vec{F}:X\subset\mathbb{R}^n\rightarrow\mathbb{R}n$.

If $\vec{F}$ is a continuous function, then we say that $\vec{F}$ is a \emph{continuous vector field}.
\end{definition}

In this definition, we're thinking about the inputs as points and the outputs as vectors, even though both are in $\mathbb{R}^n$.

ADD EXAMPLES

\section*{Graphing and Scaling}

If the vectors in our vector field are particularly long, graphing our vector field can quickly turn into a cluttered mess. In these situations, it can be useful to scale the vectors, so that we can more clearly see the behavior of our vector field.

EXAMPLE

In fact, most graphing software will automatically scale vector fields, whether you want them to or not.



\end{document}