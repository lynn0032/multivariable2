\documentclass{ximera}  

\title{Triple Integrals over Elementary Regions}  
\author{Melissa Lynn}
\outcome{Evaluate triple integrals over elementary regions, and change the order of integration. Focus on visualizing regions.}

\begin{document}  
\begin{abstract}  
\end{abstract}  
\maketitle  

In this section, we'll look at how we can evaluate triple integrals over elementary regions. This will be similar to integrating double integrals over elementary regions, so we'll start with a quick review.

\section*{Review of double integrals over elementary regions}

To define a double integral over a region $D$, we used an extension function in order to leverage our definition of double integrals over rectangles. This extension function was defined as
\[
f^{\text{ext}}(x,y) = \begin{cases}
f(x,y), & (x,y)\in D\\
0, & (x,y)\notin D
\end{cases}.
\]
We were then able to define the double integral of $f$ over $D$ using a double integral of $f^{\text{ext}}$ over a rectangle containing $D$.

\begin{definition}
Let $D$ be a bounded region in $\mathbb{R}^2$, and let $R$ be a rectangle containing $D$. Let $f:X\subset\mathbb{R}^2\rightarrow\mathbb{R}$ be a function defined on $D$. Then we define the \emph{double integral of $f$ over $D$} to be
\[
\iint_D f\;dA = \iint_R f^{\text{ext}}\;dA.
\]
\end{definition}

We describe elementary regions in $\mathbb{R}^2$ using inequalities, which then provide our bounds of integration.

\begin{proposition}
Consider an $x$-simple region $D$, which can be described as the set of points $(x,y)$ such that
\begin{align*}
a\leq &x\leq b\text{, and}\\
g(x)\leq &y \leq h(x),
\end{align*}
where $g(x)$ and $h(x)$ are continuous functions. Then we can evaluate the double integral of a function $f(x,y)$ over $D$ as
\[
\iint_D f(x,y)\;dA = \int_a^b\int_{g(x)}^{h(x)} f(x,y)\;dydx.
\]

Consider a $y$-simple region $D$, which can be described as the set of points $(x,y)$ such that
\begin{align*}
c\leq & y\leq d\text{, and}\\
g(y)\leq & x \leq h(y),
\end{align*}
where $g(y)$ and $h(y)$ are continuous functions. Then we can evaluate the double integral of a function $f(x,y)$ over $D$ as
\[
\iint_D f(x,y)\;dA = \int_c^d\int_{g(y)}^{h(y)} f(x,y)\;dxdy.
\]
\end{proposition}

When we wished to change the order of integration, it was important to visualize the region of integration, in order to correctly transform the inequalities.

\section*{Triple integrals over elementary regions}

We'll now define the triple integral of a function $f$ over a bounded region $D$ in $\mathbb{R}^3$.

\begin{definition}
Let $f:D\subset\mathbb{R}^3\rightarrow\mathbb{R}$ be a function defined on a bounded region $D$ in $\mathbb{R}^3$. We define an extension function by
\[
f^{\text{ext}}(x,y,z) = \begin{cases}
f(x,y,z), & (x,y,z)\in D\\
0, & (x,y,z)\notin D
\end{cases}.
\]
Let $B$ be a box containing $D$. We define the \emph{triple integral of $f$ over $D$} to be
\[
\iiint_D f\;dV = \iiint_B f^{\text{ext}}\;dV.
\]
\end{definition}

As with double integrals, we could most easily evaluate integrals over elementary regions, which we now define in $\mathbb{R}^3$.

\begin{definition}
An \emph{elementary region} $D$ in $\mathbb{R}^3$ is the set of points $(x,y,z)$ which can be described as one of the following:
\begin{itemize}
\item $a\leq x\leq b$, $f_1(x)\leq y \leq f_2(x)$, and $g_1(x,y)\leq z\leq g_2(x,y)$, where $f_1$, $f_2$, $g_1$ and $g_2$ are continuous functions. In this case,
\[
\iiint_D f\;dv = \int_a^b\int_{f_1(x)}^{f_2(x)}\int_{g_1(x,y)}^{g_2(x,y)}f\;dzdydx.
\]
\item $a\leq x\leq b$, $f_1(x)\leq z \leq f_2(x)$, and $g_1(x,z)\leq y\leq g_2(x,z)$, where $f_1$, $f_2$, $g_1$ and $g_2$ are continuous functions. In this case,
\[
\iiint_D f\;dv = \int_a^b\int_{f_1(x)}^{f_2(x)}\int_{g_1(x,z)}^{g_2(x,z)}f\;dydzdx.
\]
\item $a\leq y\leq b$, $f_1(y)\leq x \leq f_2(y)$, and $g_1(x,y)\leq z\leq g_2(x,y)$, where $f_1$, $f_2$, $g_1$ and $g_2$ are continuous functions. In this case,
\[
\iiint_D f\;dv = \int_a^b\int_{f_1(y)}^{f_2(y)}\int_{g_1(x,y)}^{g_2(x,y)}f\;dzdxdy.
\]
\item $a\leq y\leq b$, $f_1(y)\leq z \leq f_2(y)$, and $g_1(y,z)\leq x\leq g_2(y,z)$, where $f_1$, $f_2$, $g_1$ and $g_2$ are continuous functions. In this case,
\[
\iiint_D f\;dv = \int_a^b\int_{f_1(y)}^{f_2(y)}\int_{g_1(y,z)}^{g_2(y,z)}f\;dxdzdy.
\]
\item $a\leq z\leq b$, $f_1(z)\leq x \leq f_2(z)$, and $g_1(x,z)\leq y\leq g_2(x,z)$, where $f_1$, $f_2$, $g_1$ and $g_2$ are continuous functions. In this case,
\[
\iiint_D f\;dv = \int_a^b\int_{f_1(z)}^{f_2(z)}\int_{g_1(x,z)}^{g_2(x,z)}f\;dydxdz.
\]
\item $a\leq z\leq b$, $f_1(z)\leq y \leq f_2(z)$, and $g_1(y,z)\leq x\leq g_2(y,z)$, where $f_1$, $f_2$, $g_1$ and $g_2$ are continuous functions. In this case,
\[
\iiint_D f\;dv = \int_a^b\int_{f_1(z)}^{f_2(z)}\int_{g_1(y,z)}^{g_2(y,z)}f\;dxdydz.
\]
\end{itemize}
\end{definition}

Since there are so many different types of elementary regions in $\mathbb{R}^3$, we will not bother to distinguish between them. 

We'll now look at an example of evaluating triple integrals over elementary regions in $\mathbb{R}^3$.

\begin{example}
Consider the region $D$ in $\mathbb{R}^3$ bounded by the surfaces $x=1$, $y=-x^2$, $y=x^2$, $z = -1$, and $z=xy^2$, pictured below.

PICTURE

This region can be described with the inequalities
\begin{align*}
0\leq &x\leq 1,\\
-x^2\leq &y\leq x^2,\\
-1\leq &z\leq xy^2.
\end{align*}
So, we can integrate the function $f(x,y,z) = xz$ over $D$ as
\begin{align*}
\iiint_D f\;dV &= \int_0^1\int_{-x^2}^{x^2}\int_{-1}^{xy^2} xz\;dzdydx\\
&= \int_0^1\int_{-x^2}^{x^2}\left(\frac{1}{2}xz^2\right)_{z = -1}^{z = xy^2}\;dydx\\
&= \int_0^1\int_{-x^2}^{x^2}\left(\frac{1}{2}x^3y^4-\frac{1}{2}x\right)\;dydx\\
&= \int_0^1 \left(\frac{1}{10}x^3y^5 - \frac{1}{2}xy\right)_{y=-x^2}^{y=x^2}\;dx\\
&= \int_0^1 \left(\left(\frac{1}{10}x^{13} - \frac{1}{4}x^3\right) - \left(-\frac{1}{10}x^{13} + \frac{1}{4}x^3\right)\right)\;dx\\
&= \int_0^1\left(\frac{1}{5}x^{13} - \frac{1}{2}x^3\right)\;dx\\
&= \left(\frac{1}{70}x^{14} - \frac{1}{8}x^4\right)_{x = 0}^{x=1}\\
&= \frac{1}{70} - \frac{1}{8}\\
&= -\frac{31}{280}.
\end{align*}
\end{example}

\section*{Changing the order of integration}

Now let's look at how we can change the order of integration in a triple integral. The most challenging part of this process is accurately visualizing the domain of integration from different perspectives. It can be helpful to think of the ``shadow'' that the region casts on the $xy$-, $xz$-, and $yz$-planes, in order to more effectively visualize the region.

\begin{example}
Consider the region $D$ in the first octant, bounded by the plane $x+2y+3z = 6$. This region is pictured below, along with its projections onto the $xy$-, $xz$-, and $yz$-planes. 

PICTURE

We'll set up integrals over this region, with three different orders of integration.

First, let's set up a $dzdydx$ integral. Here, it's helpful to look at the projection of $D$ onto the $xy$-plane.

PICTURE

We see that this projection can be described with the inequalities
\begin{align*}
0\leq &x\leq 6,\\
0\leq &y\leq \frac{6-x}{2}.
\end{align*}

Then, we also have $0\leq z\leq \frac{6-2y-x}{3}$. So we can set up an integral
\[
\iiint_D f\;dV = \int_0^6\int_0^{(6-x)/2}\int_0^{(6-2y-x)/3} f\;dzdydx.
\]

Next, let's set up a $dzdxdy$ integral. Again, we'll look at the projection of $D$ onto the $xy$-plane. 

PICTURE

From a different perspective, this projection can be described with the inequalities
\begin{align*}
0\leq &y\leq 3,\\
0\leq &x\leq 6-2y.
\end{align*}

We also have $0\leq z\leq \frac{6-2y-x}{3}$, so we can set up an integral
\[
\iiint_D f\;dV = \int_0^3\int_0^{6-2y}\int_0^{(6-2y-x)/3} f\;dzdxdy.
\]

Finally, we'll set up a $dxdydz$ integral. Here, it will be helpful to look at the projection of $D$ onto the $yz$-plane.

PICTURE

This projection can be described with the inequalities
\begin{align*}
0\leq &y\leq 3,\\
0\leq &z\leq \frac{6-2y}{3}.
\end{align*}

We also have $0\leq x\leq 6-2y-3z$, so we can set up an integral
\[
\iiint_D f\;dV = \int_0^3\int_0^{(6-2y)/3}\int_0^{6-2y-3z} f\;dxdydz.
\]
\end{example}

\begin{example}
Consider the integral
\[
\int_1^3\int_{-x}^0\int_0^{x+y} e^{x^2}y\;dzdydx.
\]
We'll change the order of integration to $dxdydz$. To do this, we first need to figure out what the domain of integration looks like. From the bounds on $x$ and $y$, we have the inequalities
\begin{align*}
1\leq &x\leq 3,\\
-x\leq y\leq 0.
\end{align*}
These determine the following region in the $xy$-plane.

PICTURE

Putting this together with the bounds $0\leq z\leq x+y$, we can visualize our region of integration in $\mathbb{R}^3$.

PICTURE

Below, we have the projection of this region onto the $yz$-plane.

PICTURE

We can describe the projection with the inequalities
\begin{align*}
-3\leq &y\leq 0,\\
0\leq &z\leq y+3.\\
\end{align*}
Over this projection, the $x$-coordinates in our region are bounded by the planes $z=x+y$ and $x=3$. In inequalities, this can be written
\[
y-z\leq x\leq 3.
\]
So, we have
\[
\int_1^3\int_{-x}^0\int_0^{x+y} e^{x^2}y\;dzdydx = \int_{-3}^0\int_0^{y+3}\int_{y-z}^3 e^{x^2}y\;dxdydz
\]

\end{example}

\section*{Volume of an elementary region}

When working with double integrals, we discovered that we could find the area of a region in $\mathbb{R}^2$ by integrating $1$ over the region. That is,
\[
\text{area}(D) = \iint_D 1\;dA.
\]

Similarly, we can find the volume of a region in $\mathbb{R}^3$ by integrating $1$ over the region.

\begin{proposition}
If $D$ is an elementary region in $\mathbb{R}^3$, then
\[
\text{volume}(D) = \iiint_D 1\;dV.
\]
\end{proposition}

Let's look at an example of using this to compute volume.

\begin{example}
Let $D$ be the region in the first octant bounded by the plane $x+y+z=1$. We'll find the volume of this region.

PICTURE

The region $D$ can be described with the inequalities
\begin{align*}
0\leq x\leq 1,\\
0\leq y\leq 1-x,\\
0\leq z\leq 1-x-y.
\end{align*}
So we have
\begin{align*}
\text{volume}(D) = \int_0^1\int_0^{1-x}\int_0^{1-x-y} 1\;dzdydx\\
&= \int_0^1\int_0^{1-x}(z)_0^{1-x-y} \;dydx\\
&= \int_0^1\int_0^{1-x}1-x-y \;dydx\\
&= \int_0^1\left(y-xy-\frac{1}{2}y^2\right)_0^{1-x} \;dx\\
&= \int_0^1\left((1-x)-x(1-x)-\frac{1}{2}(1-x)^2\right) \;dx\\
&= \int_0^1\left(1-x-x+x^2-\frac{1}{2}+x-\frac{1}{2}x^2\right) \;dx\\
&= \int_0^1\left(\frac{1}{2}-x+\frac{1}{2}x^2\right) \;dx\\
&= \left(\frac{1}{2}x-\frac{1}{2}x^2+\frac{1}{6}x^3\right)_{x=0}^{x=1}\\
&= \frac{1}{6}.
\end{align*}
Thus, the volume of the region $D$ is $\frac{1}{6}$.
\end{example}


\end{document}