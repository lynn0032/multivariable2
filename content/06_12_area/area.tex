\documentclass{ximera}  

\title{Area of a Region}  
\author{Melissa Lynn}
\outcome{Use Green's theorem to compute the area of a region in the plane.}

\begin{document}  
\begin{abstract}  
\end{abstract}  
\maketitle  

We've seen how Green's theorem relates a vector line integral over the boundary of a region to a double integral over the region.

\begin{theorem}
\textbf{Green's Theorem.} Let $D$ be a closed an bounded region in $\mathbb{R}^2$, whose boundary $\partial D$ consists of finitely many simple and piecewise smooth curves. Let $\vec{F}$ be a $\mathcal{C}^1$ vector field defined on $D$, written in components as $\vec{F}(x,y) = (M(x,y), N(x,y))$. Then
\[
\oint_{\partial D}\vec{F}\cdot d\vec{s} = \iint_D \left(\frac{\partial N}{\partial x} - \frac{\partial M}{\partial y}\right)\;dA.
\]
\end{theorem}

We've used Green's theorem to more easily compute some vector line integrals, and in this section, we'll see how Green's theorem can be used to find the area of a region in $\mathbb{R}^2$.

\section*{Area of a region}

Suppose we wish to find the area of a region $D$ in $\mathbb{R}^2$.

PICTURE

To find this area, we can instead find a volume with the same value. That is, we can create a solid of height $1$ over the region, and find the volume of this solid.

PICTURE

We can use a double integral to compute this volume, giving us the area of the region $D$.

\begin{proposition}
Let $D$ be a region in $\mathbb{R}^2$. Then
\[
\text{area of D} = \iint_D 1\;dA.
\]
\end{proposition}

We'll use this fact to find the area of the unit circle. From geometry, we expect this area to be $\pi$.

\begin{example}
Let $D$ be the unit circle in $\mathbb{R}^2$. To find the area of $D$, we use
\[
\text{area of D} = \iint_D 1\;dA.
\]
In polar coordinates, the unit circle can be described with the inequalities
\begin{align*}
0\leq r\leq 1,\\
0\leq \theta\leq 2\pi.
\end{align*}
Changing to polar coordinates to evaluate the double integral, we have
\begin{align*}
\iint_D 1\;dA &= \int_0^{2\pi} \int_0^1 r\;drd\theta\\
&= \int_0^{2\pi} \frac{1}{2}r^2|_0^1\;d\theta\\
&= \int_0^{2\pi} \frac{1}{2}\;d\theta\\
&= \pi.
\end{align*}
So, we have confirmed that the area of the unit circle is $\pi$.
\end{example}

\section*{Area using Green's theorem}

Now, let's see how we can use Green's theorem to find the area of a region. Suppose we have a curve $C$ enclosing a region $D$, satisfying the hypotheses for Green's theorem.

PICTURE

Now, suppose we have a vector field $\vec{F}(x,y) = (M(x,y), N(x,y))$ such that $\frac{\partial N}{\partial x} - \frac{\partial M}{\partial y} = 1$. There are many possible choices for such a vector field; examples include
\begin{align*}
\vec{F}(x,y) &= (x,x),\\
\vec{F}(x,y) &= (-y/2, x/2),\\
\vec{F}(x,y) &= \left(y+\sin(x), e^{y^2}\right).
\end{align*}

Then, Green's theorem gives us
\[
\oint_{C}\vec{F}\cdot d\vec{s} = \iint_D 1\;dA.
\]
So, we can find the area of $D$ by integrating the vector field $F$ over the boundary of $D$.

Let's look at an example to see this in action.

\begin{example}
We'll find the area enclosed by an ellipse $C$, given by the equation $\frac{x^2}{a^2} + \frac{y^2}{b^2} = 1$.

PICTURE

In this example, we'll use the vector field $\vec{F}(x,y) = (-y/2, x/2)$ to compute this area. By Green's theorem, we have
\begin{align*}
\text{Area} &= \iint_D 1\;dA\\
&= \oint_{C}\vec{F}\cdot d\vec{s}.
\end{align*}
We can parametrize the ellipse as $\vec{x}(t) = (a\cos t, b\sin t)$ for $0\leq t\leq 2\pi$. Notice that this parametrization gives the correct orientation for the ellipse. Now, we evaluate our line integral.
\begin{align*}
\oint_{C}\vec{F}\cdot d\vec{s} &= \int_0^{2\pi} \vec{F}(a\cos t, b\sin t)\cdot (-a\sin t, b\cos t)dt\\
&= \int_0^{2\pi} \left(-\frac{b}{2}\sin t, \frac{a}{2}\cos t\right)\cdot (-a\sin t, b\cos t)dt\\
&= \int_0^{2\pi} \left(\frac{ab}{2}\sin^2 t + \frac{ab}{2}\cos^2 t\right)dt\\
&= \int_0^{2\pi} \left(\frac{ab}{2}\right)dt\\
&= ab\pi
\end{align*}
Thus, the area enclosed by the ellipse $\frac{x^2}{a^2} + \frac{y^2}{b^2} = 1$ is $ab\pi$.
\end{example}

\end{document}