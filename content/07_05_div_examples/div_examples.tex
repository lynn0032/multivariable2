\documentclass{ximera}  

\title{Further Examples of Divergence}  
\author{Melissa Lynn}
\outcome{Explore further examples of computing and understanding the geometry of divergence.}

\begin{document}  
\begin{abstract}  
\end{abstract}  
\maketitle  

We've defined the divergence of a vector field, and seen how this can be used as a measure of local expansion or contraction.

\begin{definition}
Let $\vec{F}:\mathbb{R}^n\rightarrow\mathbb{R}^n$ be a differentiable vector field. Then the \emph{divergence} of $\vec{F}$ is $\nabla\cdot \vec{F}$.
\end{definition}

In this section, we'll explore additional examples of computing divergence, and interpreting our answers geometrically.

\section*{Divergence examples}

Consider the vector field $\vec{F}(x,y) = \left(\frac{x}{(x^2+y^2)^p},\frac{y}{(x^2+y^2)^p} \right)$, for various values of $p$. Graphing this for various values of $p$ gives us a sense of the behavior of the vector field.

PICTURE

Let's compute the divergence, and see what this tells us about the behavior of the vector field.
\begin{align*}
\nabla\cdot \vec{F}(x,y) &= \frac{\partial}{\partial x}\left(\frac{x}{(x^2+y^2)^p}\right) + \frac{\partial}{\partial y}\left(\frac{y}{(x^2+y^2)^p}\right) \\
&= \frac{(x^2+y^2)^p - 2px(x^2+y^2)^{p-1}x}{(x^2+y^2)^{2p}} + \frac{(x^2+y^2)^p - 2py(x^2+y^2)^{p-1}y}{(x^2+y^2)^{2p}}\\
&= \frac{(x^2+y^2) - 2px^2}{(x^2+y^2)^{p+1}} + \frac{(x^2+y^2) - 2py^2}{(x^2+y^2)^{p+1}}\\
&= \frac{2(x^2+y^2) - 2p(x^2+y^2)}{(x^2+y^2)^{p+1}}\\
&= \frac{2(1-p)}{(x^2+y^2)^p}\\
\end{align*}
From this, we can see that the sign of the divergence will depend on the value of $p$. In particular
\begin{itemize}
\item If $p<1$, then the divergence is
\begin{multipleChoice}
\choice[correct]{positive.}
\choice{negative.}
\choice{zero.}
\end{multipleChoice}
\item If $p>1$, then the divergence is
\begin{multipleChoice}
\choice{positive.}
\choice[correct]{negative.}
\choice{zero.}
\end{multipleChoice}
\item If $p=1$, then the divergence is
\begin{multipleChoice}
\choice{positive.}
\choice{negative.}
\choice[correct]{zero.}
\end{multipleChoice}
\end{itemize}
Let's look at how this manifests itself in the graph of $\vec{F}$, with the local expansion or contraction of the vector field.

For $p=\frac{1}{2}$, we graph the vector field $\vec{F}(x,y) = \left(\frac{x}{(x^2+y^2)^{1/2}},\frac{y}{(x^2+y^2)^{1/2}} \right)$.

PICTURE

Here, if we look at a tiny ball around a point, we see that there are more vectors exiting the ball than coming into it, and the vectors exiting the ball are longer, so we have local expansion. This matches with the divergence being positive.

For $p = 1$, we graph the vector field $\vec{F}(x,y) = \left(\frac{x}{(x^2+y^2)},\frac{y}{(x^2+y^2)} \right)$.

PICTURE

Here, if we look at a tiny ball around a point, we see that there are more vectors exiting the ball than coming into it. However, the vectors exiting the ball are shorter than the vectors entering the ball. Because of this, it's difficult to estimate the sign of the divergence from the graph. As it turns out, there is perfect cancellation, so the divergence of the vector field is zero.

For $p=2$, we graph the vector field $\vec{F}(x,y) = \left(\frac{x}{(x^2+y^2)^2},\frac{y}{(x^2+y^2)^2} \right)$.

PICTURE

Here, if we look at a tiny ball around a point, we see that there are more vectors exiting the ball than coming into it. However, the vectors exiting the ball are shorter than the vectors entering the ball. Because of this, it's difficult to estimate the sign of the divergence from the graph. In this case, the longer vectors entering the ball more than cancel out the shorter vectors exiting the ball, and the divergence of the vector field turns out to be negative.

\end{document}