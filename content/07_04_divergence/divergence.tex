\documentclass{ximera}  

\title{Divergence}  
\author{Melissa Lynn}
\outcome{Understand the definition and geometric intepretation of divergence.}

\begin{document}  
\begin{abstract}  
\end{abstract}  
\maketitle  

We've seen how the curl of a vector field measures local rotation, and how we can compute the curl of a vector field as $\nabla\times \vec{F}$.

PICTURE

We'll now look at another local property of a vector field: local expansion or contraction. That is, how can we measure if a vector field is expanding or contracting near a point?

Imagine that we take a tiny ball around a point. If we look at the number and length of vectors entering and leaving this ball, we can decide if there is local expansion or contraction.

In the pictures below, we have some example of local expansion and contraction.

PICTURE

In this section, we'll see how we can compute this local expansion or contraction, similarly to how we computed curl.

\section*{Divergence of a vector field}

Consider a two-dimensional vector field $\vec{F}:\mathbb{R}^2\rightarrow\mathbb{R}^2$, pictured below. Write $\vec{F}(x,y) = (M(x,y), N(x,y))$.

PICTURE

At the point $P$, we can see that there is local expansion.

Focusing on the change in the vector field as we move in the positive $x$-direction near $P$, we see that the $x$-coordinates of the vectors are increasing. This means that $\frac{\partial M}{\partial x}$ is positive.

Focusing on the change in the vector field as we move in the positive $y$-direction near $P$, we can see that the $y$-coordinates of the vectors are increasing. This means that $\frac{\partial N}{\partial y}$ is positive.

When we take the sum $\frac{\partial M}{\partial x} + \frac{\partial N}{\partial y}$, this gives us a measure of the local expansion of the vector field. Let's look at how we can rewrite this using the operator $\nabla$ and a dot product.
\begin{align*}
\frac{\partial M(x,y)}{\partial x} + \frac{\partial N(x,y)}{\partial y} &= \left(\frac{\partial}{\partial x}, \frac{\partial}{\partial y}\right)\cdot (M(x,y), N(x,y))\\
&= \nabla \cdot \vec{F}(x,y).
\end{align*}

This leads us to the definition of divergence, for an $n$-dimensional vector field.

\begin{definition}
Let $\vec{F}:\mathbb{R}^n\rightarrow\mathbb{R}^n$ be a differentiable vector field. Then the \emph{divergence} of $\vec{F}$ is $\nabla\cdot \vec{F}$.
\end{definition}

\begin{example}
Consider the vector field $\vec{F}(x,y,z) = (x^2+yz, y^2z, x^2y^3z^4)$ in $\mathbb{R}^3$. We'll compute the divergence of this vector field at the the point $(1,2,3)$.
\begin{align*}
\nabla\cdot \vec{F}(x,y,z) &= \frac{\partial}{\partial x}(x^2+yz) + \frac{\partial}{\partial y}(y^2z) + \frac{\partial}{\partial z}(x^2y^3z^4)\\
&= 2x + 2yz + 4x^2y^3z^3
\end{align*}
Evaluating at the point $(1,2,3)$, we have
\begin{align*}
\nabla\cdot \vec{F}(1,2,3) &= 2\cdot 1 + 2\cdot 2\cdot 3 + 4\cdot 1^2\cdot 2^3\cdot 4^3\\
&= 2062.
\end{align*}
\end{example}

\section*{Geometric interpretation of divergence}

Recall that divergence of a vector field measures the local expansion or contraction of a vector field near a point. With this in mind, for each of the vector fields below, estimate whether the divergence of $F$ at $P$ is positive, negative, or zero.

\begin{example}
PICTURE

\begin{multipleChoice}
\choice[correct]{positive}
\choice{negative}
\choice{zero}
\end{multipleChoice}
\end{example}

\begin{example}
PICTURE

\begin{multipleChoice}
\choice{positive}
\choice{negative}
\choice[correct]{zero}
\end{multipleChoice}
\end{example}

\begin{example}
PICTURE

\begin{multipleChoice}
\choice{positive}
\choice[correct]{negative}
\choice{zero}
\end{multipleChoice}
\end{example}



\end{document}