\documentclass{ximera}

\graphicspath{{./graphics/}}

\title{Vector Fields}
\author{Melissa Lynn}
\outcome{Understand the definition of vector fields and how to graph them.}

\begin{document}
\begin{abstract}
\end{abstract}
\maketitle

Consider a function $f:\mathbb{R}^2\rightarrow\mathbb{R}$. Such a function takes points $(x,y)$ in the plane, and assigns a real number $f(x,y)$ to each of them. There are a few ways that we can visualize these functions. Perhaps the most common way is to graph the function in $\mathbb{R}^3$. The graph consists of the points $(x,y,f(x,y))$, and we often think of the graph as being height $f(x,y)$ over the point $(x,y)$.

PICTURE

We could also visualize the graph in the plane, using a heat map to indicate the ``height'' $f(x,y)$ at points $(x,y)$.

PICTURE

We could also represent the function by writing the function values $f(x,y)$ at points $(x,y)$. Of course, we can't write these values at all points, because then we wouldn't be able to read them! But for nicely behaved functions, we can choose a representative sample of points, so that the function values at those points accurately reflect the overall behavior of the function.

PICTURE

This may remind you of a temperature map used to give the temperature across a region.

Now, suppose instead of having a \emph{value} at each point in the plane, we had a \emph{vector}.

PICTURE

This might be used to represent windspeed and direction, or the direction and strength of any force, such as gravity or a magnetic field.

Let's think about how we can translate this idea into a function. The inputs are still points in the plane, $\mathbb{R}^2$, but now the outputs are vectors, also in $\mathbb{R}^2$. Thus, we have a function $\mathbb{R}^2\rightarrow\mathbb{R}^2$. When we think of such a function as assigning vectors to points in $\mathbb{R}^2$, we call this a vector field.

\section*{Vector Fields}

We've seen that a vector field consists of vectors placed at each point in some region, and we can think of this as assigning a vector to each point in the region, and we can represent this with a function.

\begin{definition}
A \emph{vector field} is a function $\vec{F}:X\subset\mathbb{R}^n\rightarrow\mathbb{R}n$.

If $\vec{F}$ is a continuous function, then we say that $\vec{F}$ is a \emph{continuous vector field}.
\end{definition}

In this definition, we're thinking about the inputs as points and the outputs as vectors, even though both are in $\mathbb{R}^n$.

\section*{Graphing and Scaling}

Let's look at how we can visualize vector fields.

\begin{example}
Consider the vector field $\vec{F}(x,y) = \left(\frac{y}{x^2+y^2}, \frac{x}{x^2+y^2}\right)$. We'll plot the vectors $\vec{F}(x,y)$ starting at points $(x,y)$ in order to graph this vector field. To do this, we'll start by evaluating this function at a few points.

\begin{center}
\begin{tabular}{|c|c|}
\hline
$(x,y)$ & $\vec{F}(x,y)$\\
\hline
$(1,0)$ & $(0,1)$\\
$(0,1)$ & $(1,0)$\\
$(1,1)$ & $(1/2, 1/2)$\\
$(2,0)$ & $(0,1/2)$\\
$(0,2)$ & $(1/2,0)$\\
$(2,1)$ & $(1/5, 2/5)$\\
$(1,2)$ & $(2/5, 1/5)$\\
$(2,2)$ & $(1/4,1/4)$\\
\hline
\end{tabular}
\end{center}

Now, we can plot these vectors, and visualize the behavior of our vector field.

PICTURE

\end{example}

If the vectors in our vector field are particularly long, graphing our vector field can quickly turn into a cluttered mess. In these situations, it can be useful to scale the vectors, so that we can more clearly see the behavior of our vector field. In fact, most graphing software will automatically scale vector fields, whether you want them to or not.

\begin{example}
Consider the vector field $\vec{G}(x,y) = (-x, y)$. We'll begin by evaluating this function at several point.

\begin{center}
\begin{tabular}{|c|c|}
\hline
$(x,y)$ & $\vec{F}(x,y)$\\
\hline
$(0,0)$ & $(0,0)$\\
$(1,0)$ & $(-1,0)$\\
$(2,0)$ & $(-2,0)$\\
$(0,1)$ & $(0,1)$\\
$(1,1)$ & $(-1,1)$\\
$(2,1)$ & $(-2,1)$\\
$(0,2)$ & $(0,2)$\\
$(1,2)$ & $(-1,2)$\\
$(2,2)$ & $(-2,2)$\\
\hline
\end{tabular}
\end{center}

We can plot these vectors to graph the vector field $\vec{G}$.

PICTURE

Because the vectors are long and overlap with each other, it's a bit difficult to get a sense of the behavior of the vector field. To address this issue, we can scale the vectors, to avoid overlap. Below, we scale the vectors by $1/4$.

PICTURE

Here, it's easier to see the behavior of the function, though it's important to remember that the exact lengths of the vectors are no longer accurate.

\end{example}



\end{document}