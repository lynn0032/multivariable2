\documentclass{ximera}  

\title{Scalar Surface Integrals}  
\author{Melissa Lynn}
\outcome{Compute and understand the geometric meaning of scalar surface integrals.}

\begin{document}  
\begin{abstract}  
\end{abstract}  
\maketitle  

We've seen how we can compute the surface area of a parametrized surface using a double integral.

\begin{proposition}
Let $\vec{X}:D\subset\mathbb{R}^2\rightarrow\mathbb{R}^3$ be a parametric surface in $\mathbb{R}^3$. Then the surface area of $\vec{X}$ is given by
\[
\text{area}(\vec{X}) = \iint_D\|\vec{X}_s\times\vec{X}_t\|\;dsdt.
\]
\end{proposition}

We found this by looking at how grid curves divided the surface up into tiny curved rectangles, and then approximating the areas of these rectangles, and adding them up.

PICTURE

Now, suppose we have some thin surface, and we want to find its mass. If the density of the surface is constant, this is simple - we multiply the density by the surface area, and we're done. For example, if we have a hollow sphere of radius $1$ cm and density $3 \text{ kg/cm}^2$, then the sphere has surface area $\frac{4}{3}\pi \text{ cm}^2$, and mass
\[
3\cdot \frac{4}{3}\pi  = 4\pi\text{ kg}.
\]
But what if the density of the sphere isn't constant, and instead varies across the surface? That is, if the density is given by some function $\rho:S\rightarrow\mathbb{R}$ defined on the sphere $S$, how can we find the total mass across the entire surface?

More generally, how do we integrate a scalar-valued function over a surface? This brings us to scalar surface integrals.

\section*{Scalar surface integrals}

Let's think back to how we found the surface area of a parametrized surface, $\vec{X}:D\rightarrow\mathbb{R}^3$. We did this by approximating the area of curvy rectangles along the surface, with $\|\vec{X}_s\times\vec{X}_t\|$.

PICTURE

Now, suppose the density across this curvy rectangle is a constant, $\rho$. Then the mass of the curvy rectangle is $\rho\|\vec{X}_s\times\vec{X}_t\|$. Even if the density is not constant, but is given by some function $\rho(\vec{X})$ that takes points on the surface and returns the density, we can still approximate the mass using the density at a sample point, $\vec{X}^*$, in the curvy rectangle.

PICTURE

That is, the mass of the curvy rectangle will be approximately $\rho(\vec{X}^*) \rho\|\vec{X}_s\times\vec{X}_t\|$. When we add up the masses of curvy rectangles across the entire surface, and take the limit as the rectangles get arbitrarily small, we obtain a double integral,
\[
\iint_S \rho(\vec{X}(s,t)\|\vec{X}_s\times\vec{X}_t\|\;dsdt.
\]
Thus, we arrive at the definition of a scalar surface integral.

\begin{definition}
Let $\vec{X}:D\subset \mathbb{R}^2\rightarrow\mathbb{R}^3$ be a parametrization for a surface $S$ in $\mathbb{R}^3$. Let $f:S\subset \mathbb{R}^3\rightarrow\mathbb{R}$ be a continuous function which is defined on the surface $S$. We define the \emph{surface integral} of $f$ over $\vec{X}$ to be
\[
\iint_{\vec{X}} f\;dS = \iint_D f(\vec{X}(s,t))\|\vec{X}_s\times\vec{X}_t\|\;dsdt.
\]
\end{definition}

\section*{Scalar surface integral computations}

Let's look at some examples of computing scalar surface integrals.

\begin{example}
Let's integrate the function $f(x,y,z) = z^2$ over the sphere of radius $2$ centered at the origin, which can be parametrized as
\[
\vec{X}(s,t) = (2\cos s\sin t, 2\sin s \sin t, 2\cos t),
\]
for $ 0 \leq s\leq 2\pi$ and $0\leq t\leq \pi$.

First, let's find the derivatives $\vec{X}_s$ and $\vec{X}_t$.
\begin{align*}
\vec{X}_s(s,t) &= (-2\sin s\sin t, 2\cos s\sin t, 0)
\vec{X}_t(s,y) &= (2\cos s\cos t, 2\sin s\cos t, -2\sin t)
\end{align*}
Next, we'll find the cross product, $\vec{X}_s\times \vec{X}_t$.
\begin{align*}
\vec{X}_s(s,t)\times \vec{X}_t(s,t) &= (-2\sin s\sin t, 2\cos s\sin t, 0)\times (2\cos s\cos t, 2\sin s\cos t, -2\sin t) \\
&= \text{det}\begin{pmatrix}
\vec{i} & \vec{j} & \vec{k}\\
-2\sin s\sin t & 2\cos s\sin t & 0\\
2\cos s\cos t & 2\sin s\cos t & -2\sin t
\end{pmatrix}\\
&= (2\cos s\sin t)(-2\sin t)\vec{i} - (-2\sin s\sin t)(-2\sin t)\vec{j} + (-2\sin s\sin t)(2\sin s\cos t)\vec{k} - (2\cos s\sin t)(2\cos s\cos t)\vec{k}\\
&= (-4\cos s\sin^2t,-4\sin s\sin^2 t,-4\sin^2s\cos t\sin t - 4\cos^2 s\cos t\sin t)\\
&= (-4\cos s\sin^2t,-4\sin s\sin^2 t, -4\cos t\sin t)
\end{align*}
The length of this vector is
\begin{align*}
\|\vec{X}_s(s,t)\times \vec{X}_t(s,t)\| &= \left\|(-4\cos s\sin^2t,-4\sin s\sin^2 t, -4\cos t\sin t)\right\|\\
&= \sqrt{16\cos^2 s\sin^4t, 16\sin^2 s\sin^4 t, 16\cos^2 t\sin^2 t}\\
&= \sqrt{16\sin^4t + 16\cos^2 t\sin^2 t}\\
&= \sqrt{16\sin^2 t}\\
&= 4|\sin t|
\end{align*}
Since  $0\leq t \leq \pi$, this is $4\sin t$. 

Now, we evaluate the scalar surface integral.
\begin{align*}
\iint_{\vec{X}} f\;dS &= \iint_D f(\vec{X}(s,t))\|\vec{X}_s\times\vec{X}_t\|\;dsdt\\
&= \int_0^{2\pi}\int_0^{\pi}f(2\cos s\sin t, 2\sin s \sin t, 2\cos t)\cdot 4\sin t\;dtds\\
&= \int_0^{2\pi}\int_0^{\pi}4\cos^2 t)\cdot 4\sin t\;dtds\\
&= \int_0^{2\pi}\left(-\frac{16}{3}\cos^3t\right)_{t = 0}^{t=\pi}\;ds\\
&= \int_0^{2\pi}(16/3 + 16/3)\;ds\\
&= \frac{64}{3}\pi
\end{align*}
So, we have $\iint_{\vec{X}} f\;dS = \frac{64}{3}\pi$.
\end{example}

\begin{example}
Suppose the graph of the function $g(x,y) = xy$ for $0\leq x\leq 2$ and $0\leq y\leq 2$ represents a small (very) snowy region, and the depth of the snow at a point $(x,y,xy)$ is given by $4xy$, where $x$ and $y$ are in meters. Suppose also that we wish to find the total amount of snow over the region.

First, we need to figure out how to set this problem up using a scalar surface integral. Our surface is the graph of the function $g(x,y) = xy$, and we can parametrize this surface as
\[
\vec{X}(s,t) = (s,t,st)
\]
for $0\leq s\leq 2$ and $0\leq t\leq 2$. If we define the function $f(x,y,z)=4xy$, then at a point $(s,t,st)$ on the surface, we have $f(\vec{X}(s,t)) = 4st$. So, with this set up, we wish to evaluate the integral
\[
\iint{\vec{X}}f\;dS.
\]
First, we find the partial derivatives $\vec{X}_s$ and $\vec{X}_t$.
\begin{align*}
\vec{X}_s(s,t) &= (1, 0, t)\\
\vec{X}_t(s,t) &= (0,1,s)
\end{align*}
We take the cross product of these vectors, to obtain
\begin{align*}
\vec{X}_s(s,t)\times\vec{X}_t(s,t) &= \text{det}\begin{pmatrix}
\vec{i} & \vec{j} & \vec{k}\\
1 & 0 & t\\
0 & 1 & s
\end{pmatrix}\\
&= (-t)\vec{i} + (-s)\vec{j} + \vec{k}.
\end{align*}
The length of this vector is
\[
\|\vec{X}_s(s,t)\times\vec{X}_t(s,t)\| = \sqrt{t^2+s^2+1}.
\]
\end{example}
Now, we evaluate our scalar surface integral.
\begin{align*}
\iint{\vec{X}}f\;dS &= \int_0^2\int_0^2 f(s,t,st)\sqrt{t^2+s^2+1}\; dtds\\
&= \int_0^2\int_0^2 4st\sqrt{t^2+s^2+1}\; dtds\\
&= \int_0^2\left(2s\sqrt{t^2+s^2+1}^3\right)_{t = 0}^{t = 2}\;ds\\
&= \int_0^2 \frac{4}{3}s\sqrt{s^2 + 5}^3\;ds\\
&= \left(\frac{4}{15}\sqrt{s^2+5}^5\right)_{s = 0}^{s = 2}\\
&= \frac{324}{5}
\end{align*}

So, the total volume of snow is $324/5\text{ m}^3$.

\end{document}