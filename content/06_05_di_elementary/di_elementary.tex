\documentclass{ximera}  

\title{Double Integrals over Elementary Regions}  
\author{Melissa Lynn}
\outcome{Set up and evaluate double integrals over elementary regions.}

\begin{document}  
\begin{abstract}  
\end{abstract}  
\maketitle  

We've defined double integrals over rectangles, by approximating the volume under a surface with boxes.

PICTURE

\begin{definition}
Let $f:R\subset\mathbb{R}^2\rightarrow\mathbb{R}$ be a function defined on a rectangle $R$ in $\mathbb{R}^2$. The \emph{double integral} of $f$ over $R$ is
\[
\iint_R f(x,y) dA = \lim_{n\rightarrow \infty} \sum_{i,j} f(\vec{c}_{ij})\Delta x\Delta y,
\]
provided this limit exists.
\end{definition}

We used Fubini's theorem to show that we can evaluate double integrals using iterated integrals.

Now, we would like to define and evaluate double integrals over regions which aren't rectangles. We will start by defining these double integrals.

\section*{Double integrals over an arbitrary region}

Suppose we wish to integrate a function $f:X\subset\mathbb{R}^2\rightarrow\mathbb{R}$ over a bounded region $D\subset X$ contained in the domain of $f$. This region, $D$, may be an elementary region, but this is not required for the definition.

PICTURE

We will define the double integral of $f$ over $D$, using a double integral over a rectangle. That is, consider a function $f^{\text{ext}}$, which is defined to be equal to $f$ on $D$, and zero elsewhere.

\begin{definition}
Let $f^{\text{ext}}:\mathbb{R}^2\rightarrow\mathbb{R}$ be the function defined by
\[
f^{\text{ext}}(x,y) = \begin{cases}
f(x,y), & (x,y)\in D\\
0, & (x,y)\notin D
\end{cases}.
\]
\end{definition}

Now, let $R$ be a rectangle containing $D$. Since $D$ is bounded, such a rectangle exists. Then, we can see that the volume under $f$ and over $D$ is equivalent to the the volume under $f^{\text{ext}}$ and over $R$.

PICTURE

Thus, we define the double integral of $f$ over $D$ using the double integral of $f^{\text{ext}}$ over $R$.

\begin{definition}
Let $D$ be a bounded region in $\mathbb{R}^2$, and let $R$ be a rectangle containing $D$. Let $f:X\subset\mathbb{R}^2\rightarrow\mathbb{R}$ be a function defined on $D$. Then we define the \emph{double integral of $f$ over $D$} to be
\[
\iint_D f\;dA = \iint_R f^{\text{ext}}\;dA.
\]
\end{definition}

Although this provides us with a definition for double integrals, this definition isn't very useful for evaluating double integrals. Next, we'll look at how we can evaluate double integrals over elementary regions.

\section*{Evaluating double integrals over elementary regions}

When $D$ is an elementary regions, we can describe it using inequalities, and then use these inequalities for the bounds of integration.

\begin{proposition}
Consider an $x$-simple region $D$, which can be described as the set of points $(x,y)$ such that
\begin{align*}
a\leq &x\leq b\text{, and}\\
g(x)\leq &y \leq h(x),
\end{align*}
where $g(x)$ and $h(x)$ are continuous functions. Then we can evaluate the double integral of a function $f(x,y)$ over $D$ as
\[
\iint_D f(x,y)\;dA = \int_a^b\int_{g(x)}^{h(x)} f(x,y)\;dydx.
\]

Consider a $y$-simple region $D$, which can be described as the set of points $(x,y)$ such that
\begin{align*}
c\leq & y\leq d\text{, and}\\
g(y)\leq & x \leq h(y),
\end{align*}
where $g(y)$ and $h(y)$ are continuous functions. Then we can evaluate the double integral of a function $f(x,y)$ over $D$ as
\[
\iint_D f(x,y)\;dA = \int_c^d\int_{g(y)}^{h(y)} f(x,y)\;dxdy.
\]
\end{proposition}

Let's look at how we can evaluate double integrals over elementary regions.

\begin{example}
Consider the $x$-simple region $D$ below, which is bounded by the curves $x=-1$, $x=1$, $y=0$, and $y=x^2$. 

PICTURE

Let's integrate the function $f(x,y)=x^2y$ over this region. Since $D$ can be described with the inequalities
\begin{align*}
-1\leq &x\leq 1\text{, and}\\
0\leq &y\leq x^2.
\end{align*}
We have
\begin{align*}
\iint_D f\;dA &= \int_{-1}^1\int_0^{x^2}x^2y\;dydx\\
&=\int_{-1}^1\left(\frac{1}{2}x^2y^2|_{y=0}^{y=x^2}\right)\;dx\\
&= \int_{-1}^1\left(\frac{1}{2}x^2(x^2)^2 - 0\right)\;dx\\
&= \int_{-1}^1\left(\frac{1}{2}x^6\right)\;dx\\
&= \left(\frac{1}{14}x^7\right)|_{x=-1}^{x=1}\\
&= \left(\frac{1}{14}\cdot 1^7\right) - \left(\frac{1}{14}\cdot (-1)^7\right)\\
&= \frac{1}{7}.
\end{align*}
\end{example}

When evaluating these double integrals, it can be very useful to start be drawing the region of integration, in order to determine the bounds.

\begin{example}
We'll evaluate the double integral $\iint_D f\;dA$, where $f(x,y) = \sin(y^2)$ and $D$ is the region below.

PICTURE

The region $D$ is $y$-simple, and can be described with the inequalities
\begin{align*}
0\leq &y\leq \sqrt{\pi}\text{, and}\\
-y\leq &x\leq y.
\end{align*}
We evaluate the double integral.
\begin{align*}
\iint_D f\;dA &= \int_0^{\sqrt{\pi}}\int_{-y}^y \sin(y^2)\;dxdy\\
&= \int_0^{\sqrt{\pi}}\left(\sin(y^2)x\right)|_{x={-y}}^{x=y}\;dy\\
&= \int_0^{\sqrt{\pi}}\left(2y\sin(y^2)\right)\;dy\\
&= \left(-\cos(y^2)\right)|_{y=0}^{y=\sqrt{\pi}}\\
&= -\cos(\sqrt{\pi}^2) + \cos(0^2)\\
&= 2
\end{align*}
\end{example}

\end{document}