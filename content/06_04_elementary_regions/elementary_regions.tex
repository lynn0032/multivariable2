\documentclass{ximera}  

\title{Elementary Regions}  
\author{Melissa Lynn}
\outcome{Identify and describe elementary regions.}

\begin{document}  
\begin{abstract}  
\end{abstract}  
\maketitle  

We've defined double integrals over rectangles, and we've used Fubini's theorem to convert them to iterated integrals, which are straightforward to compute. But what if we need to integrate a function over a region that isn't a rectangle?

For example, consider the region below, which is bounded by the curves $y=x^2$, $y=x^2+1$, $x=1$, and $x=2$.

PICTURE

We can describe this region as the set of points $(x,y)$ such that $1\leq x\leq 2$ and $x^2\leq y\leq x^2+1$. Here, we can describe the region by bounding the $x$-coordinate with constants, and bounding the $y$-coordinate with continuous functions of $x$. We'll soon see that this kind of description translates easily to integration, and this is our first example of an elementary region.

\section*{Elementary regions}

We give the definition of elementary regions, which will be the most natural regions to integrate over.

\begin{definition}
The following are types of \emph{elementary regions}.

Suppose a region $R$ can be described as the set of points $(x,y)$ such that
\begin{align*}
a\leq &x\leq b\text{, and}\\
f(x)\leq &y \leq g(x),
\end{align*}
where $f(x)$ and $g(x)$ are continuous functions. Then we say that $R$ is \emph{$x$-simple}.

Suppose a region $R$ can be described as the set of points $(x,y)$ such that
\begin{align*}
c\leq & y\leq d\text{, and}\\
f(y)\leq & x \leq g(y),
\end{align*}
where $f(y)$ and $g(y)$ are continuous functions. Then we say that $R$ is \emph{$y$-simple}.
\end{definition}

We'll now look at some examples of $x$-simple and $y$-simple regions. Note that some regions are both $x$-simple and $y$-simple.

\begin{example}
Consider the region $R$ below, which is bounded by the graphs of $y=x^2$ and $x=y^2$.

PICTURE

This region is both $x$-simple and $y$-simple. To see that it is $x$-simple, we can describe $R$ as the set of points $(x,y)$ such that 
\begin{align*}
0\leq &x\leq 1\text{, and}\\
x^2\leq y\leq \sqrt{x}.
\end{align*}
To see that $R$ is $y$-simple, we can describe it as the set of points $(x,y)$ such that
\begin{align*}
0\leq &y\leq 1\text{, and}\\
y^2\leq &x\leq \sqrt{y}.
\end{align*}
\end{example}

\begin{example}
Consider the region $R$ below, which is bounded by the graphs of $y=\cos x$, $y=1$, $x=-\pi$, and $x=\pi$.

PICTURE

This region is $x$-simple, since $R$ is the set of points $(x,y)$ such that
\begin{align*}
-\pi\leq &x\leq \pi\text{, and}\\
\cos x\leq &y\leq 1.
\end{align*}
The region $R$ is not $y$-simple, since any inequality $f(y)\leq x\leq g(y)$ cannot have a ``hole'' in the middle. However, $R$ is the union of two $y$-simple regions $R_1$ and $R_2$, where $R_1$ is the set of points $(x,y)$ such that
\begin{align*}
-1\leq &y\leq 1\text{, and}\\
-\pi\leq &x\leq -\arccos y,
\end{align*} 
and $R_2$ is the set of points $(x,y)$ such that
\begin{align*}
-1\leq &y\leq 1\text{, and}\\
\arccos y\leq &x\leq 1.
\end{align*} 
\end{example}


\end{document}