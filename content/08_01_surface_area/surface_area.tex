\documentclass{ximera}  

\title{Surface Area}  
\author{Melissa Lynn}
\outcome{Compute the surface area of a parametric surface.}

\begin{document}  
\begin{abstract}  
\end{abstract}  
\maketitle  

Suppose we have a parametrized surface, $\vec{X}:D\subset\mathbb{R}^2\rightarrow\mathbb{R}^3$, and suppose we wish to find the surface area of this surface.

PICTURE

Recall that by fixing one parameter, we get a coordinate curve. Doing this for several values for each parameter, we obtain a ``grid'' on the surface.

PICTURE

This grid consists of many small regions, which we'll describe as curvy rectangles. If we can approximate the area of the curvy rectangles, and add all of these areas together, we'll approximate the surface area. This provides us with the idea for our surface area computation.

To implement this, we need to begin by approximating the area of a small curvy rectangle. We can think of the curvy rectangle as the image of a rectangle in $(s,t)$-coordinates, under a transformation $\vec{X}$.

PICTURE

In $(s,t)$-coordinates, suppose that the rectangle has base $\Delta s$ and height $\Delta t$, so the area is $\Delta s\Delta t$.

We can approximate the sides of the curvy rectangle with the vectors $\vec{X}_s(s_0,t_0)\Delta s$ and $\vec{X}_t(s_0,t_0)\Delta t$.

PICTURE

This is a reasonable approximation because these vectors are tangent to the grid curves, and are scaled to the appropriate length by the rectangle in $(s,t)$-coordinates. 

Then, we can approximate the curvy rectangle with the rectangle determined by the vectors $\vec{X}_s(s_0,t_0)\Delta s$ and $\vec{X}_t(s_0,t_0)\Delta t$.

PICTURE

From linear algebra, we know that this rectangle has area 
\[
\left\|\vec{X}_s(s_0,t_0)\Delta s\times \vec{X}_t(s_0,t_0)\Delta t\right\|,
\]
and we can rewrite this as
\[
\left\|\vec{X}_s(s_0,t_0)\times \vec{X}_t(s_0,t_0)\right\|\Delta s\Delta t.
\]
Now, if we add up the areas of all of these approximations, and let $\Delta s\rightarrow 0$ and $\Delta t \rightarrow 0$ (so that the rectangles get very small), we obtain the double integral,
\[
\iint_D\|\vec{X}_s\times\vec{X}_t\|\;dsdt.
\]
This provides us with a formula to compute the surface area of a parametrized surface.

\begin{proposition}
Let $\vec{X}:D\subset\mathbb{R}^2\rightarrow\mathbb{R}^3$ be a parametric surface in $\mathbb{R}^3$. Then the surface area of $\vec{X}$ is given by
\[
\text{area}(\vec{X}) = \iint_D\|\vec{X}_s\times\vec{X}_t\|\;dsdt.
\]
\end{proposition}

\section*{Surface area computations}

We'll now use our formula for the surface area of a parametric surface to confirm several well-known surface area formulas from geometry.

\begin{example}
We'll find the surface area of a sphere of radius $R$, which can be parametrized by
\[
\vec{X}(s,t) = (R\cos s\sin t, R\sin s\sin t, R\cos t),
\]
for $0\leq t\leq \pi$ and $0\leq s\leq 2\pi$.
First, let's find the derivatives $\vec{X}_s$ and $\vec{X}_t$.
\begin{align*}
\vec{X}_s(s,t) &= \left(\frac{\partial}{\partial s} R\cos s\sin t, \frac{\partial}{\partial s} R\sin s\sin t, \frac{\partial}{\partial s}\cos t\right)\\
&= (-R\sin s\sin t, R\cos s\sin t, 0)
\vec{X}_t(s,y) &= \left(\frac{\partial}{\partial t} R\cos s\sin t, \frac{\partial}{\partial t} R\sin s\sin t, \frac{\partial}{\partial t}R\cos t\right)\\
&= (R\cos s\cos t, R\sin s\cos t, -R\sin t)
\end{align*}
Next, we'll find the cross product, $\vec{X}_s\times \vec{X}_t$.
\begin{align*}
\vec{X}_s(s,t)\times \vec{X}_t(s,t) &= (-R\sin s\sin t, R\cos s\sin t, 0)\times (R\cos s\cos t, R\sin s\cos t, -R\sin t) \\
&= \text{det}\begin{pmatrix}
\vec{i} & \vec{j} & \vec{k}\\
-R\sin s\sin t & R\cos s\sin t & 0\\
R\cos s\cos t & R\sin s\cos t & -R\sin t
\end{pmatrix}\\
&= (R\cos s\sin t)(-R\sin t)\vec{i} - (-R\sin s\sin t)(-R\sin t)\vec{j} + (-R\sin s\sin t)(R\sin s\cos t)\vec{k} - (R\cos s\sin t)(R\cos s\cos t)\vec{k}\\
&= (-R^2\cos s\sin^2t,-R^2\sin s\sin^2 t,-R^2\sin^2s\cos t\sin t - R^2\cos^2 s\cos t\sin t)\\
&= (-R^2\cos s\sin^2t,-R^2\sin s\sin^2 t, -R^2\cos t\sin t)
\end{align*}
The length of this vector is
\begin{align*}
\|\vec{X}_s(s,t)\times \vec{X}_t(s,t)\| &= \left\|(-R^2\cos s\sin^2t,-R^2\sin s\sin^2 t, -R^2\cos t\sin t)\right\|\\
&= \sqrt{R^4\cos^2 s\sin^4t, R^4\sin^2 s\sin^4 t, R^4\cos^2 t\sin^2 t}\\
&= \sqrt{R^4\sin^4t + R^4\cos^2 t\sin^2 t}\\
&= \sqrt{R^4\sin^2 t}\\
&= R^2|\sin t|
\end{align*}
Since  $0\leq t \leq \pi$, this is $R^2\sin t$. 

Then the surface area is
\begin{align*}
\text{area} &= \int_0^{\pi}\int_0^{2\pi}\|\vec{X}_s(s,t)\times \vec{X}_t(s,t)\|\;dsdt\\
&= \int_0^{\pi}\int_0^{2\pi}R^2\sin t\;ds\;dt\\
&= \int_0^{\pi}2\pi R^2\sin t\;dt\\
&= 2\pi R^2(-\cos t)_{t = 0}^{t=\pi}\\
&= 2\pi R^2(-(-1) - (-1))\\
&= 4\pi R^2.
\end{align*}
Thus, we have shown that the surface area of a sphere of radius $R$ is $4\pi R^2$.
\end{example}

\begin{example}
We'll compute the surface area of a right circular cone of radius $R$ and height $H$.

PICTURE

In order to do this, we'll need to split the surface into two pieces: the top disc, and the rest of the cone.

PICTURE

We'll start by finding the surface area of the top disc. We can parametrize the disc as
\[
\vec{X}(s,t) = (s\cos t, s\sin t, H),
\]
for $0\leq s\leq R$ and $0\leq t\leq 2\pi$. The partial derivatives are then
\begin{align*}
\vec{X}_s(s,t) = (\cos t, \sin t, 0),\\
\vec{X}_t(s,t) = (-s\sin t, s\cos t, 0).
\end{align*}
Taking the cross product, we have
\begin{align*}
\vec{X}_s(s,t)\times\vec{X}_t(s,t) &= \text{det}\begin{pmatrix}
\vec{i} & \vec{j} & \vec{k}\\
\cos t & \sin t & 0\\
-s\sin t & s\cos t & 0
\end{pmatrix}\\
&= (s\cos^2 t + s\sin^2 t)\vec{k}\\
&= s\vec{k}
\end{align*}
Since $s$ is nonnegative, the length of this vector is
\[
\|\vec{X}_s(s,t)\times\vec{X}_t(s,t)\| = s.
\]
Now, we can find the surface area of the disc.
\begin{align*}
\text{area} &= \int_0^{2\pi}\int_0^R s\;dsdt\\
&= \int_0^{2\pi} \left(\frac{s^2}{2}\right)_{s = 0}^{s = R}\;dt\\
&= \int_0^{2\pi} \frac{R^2}{2}\;dt\\
&= \pi R^2
\end{align*}
Unsurprisingly, this is the area of a circle of radius $R$.

Next, we'll find the surface area of the remaining portion of the cone. This can be parametrized as
\[
\vec{Y}(s,t) = \left(s \cos t, s \sin t, \frac{H}{R}s\right),
\]
for $0\leq s\leq R$ and $0\leq t\leq 2\pi$.

The partial derivatives are
\begin{align*}
\vec{Y}_s(s,t) &= (\cos t, \sin t, H/R),\\
\vec{Y}_t(s,t) &= (-s\sin t, s\cos t, 0).
\end{align*}
Taking the cross product of these vectors, we have
\begin{align*}
\vec{Y}_s(s,t)\times\vec{Y}_t(s,t) &= \text{det}\begin{pmatrix}
\vec{i} & \vec{j} & \vec{k} \\
\cos t & \sin t & H/R\\
-s\sin t & s\cos t & 0
\end{pmatrix}\\
&= (-(H/R) s\cos t)\vec{i} + (-(H/R)s\sin t)\vec{j} + (s\cos^2 t + s\sin^2 t)\vec{k}\\
&= (-(H/R) s\cos t)\vec{i} + (-(H/R)s\sin t)\vec{j} + s\vec{k}.
\end{align*}
The length of this vector is
\begin{align*}
\|\vec{Y}_s(s,t)\times\vec{Y}_t(s,t)\| &= \sqrt{(-(H/R) s\cos t)^2 + (-(H/R) s\sin t)^2 + s^2}\\
&= \sqrt{H^2/R^2\cdot s^2 + s^2}\\
&= s\sqrt{H^2/R^2+1},
\end{align*}
since $s$ is nonnegative.

Now, we can compute the surface area.
\begin{align*}
\text{area} &= \int_0^{2\pi}\int_0^R s\sqrt{H^2/R^2+1}\;dsdt\\
&= \sqrt{H^2/R^2+1}\int_0^{2\pi} \left(\frac{s^2}{2}\right)_{s = 0}^{s = R}\;dt\\
&= \sqrt{H^2/R^2+1}\int_0^{2\pi} \frac{R^2}{2}\;dt\\
&= \sqrt{H^2/R^2+1}\cdot \pi R^2
\end{align*}
So, the total surface area of the cone is
\[
\pi R^2 + \sqrt{H^2/R^2+1}\cdot \pi R^2 = \pi R \left(R + \sqrt{H^2 + R^2}\right).
\]
\end{example}


\end{document}