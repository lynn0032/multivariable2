\documentclass{ximera}  

\title{Change of Variables in Triple Integrals}  
\author{Melissa Lynn}
\outcome{Understand how to change coordinates in triple integrals.}

\begin{document}  
\begin{abstract}  
\end{abstract}  
\maketitle 

When working with double integrals, we found that it was sometimes advantageous to change coordinates. This was often used to make the region of integration easier to describe.

When we change from one coordinate system to another, this can change volumes and areas. To account for this, we incorporated a scaling factor into our differential.

More precisely, if we have a change of coordinates given by a function $T:\mathbb{R}^2\rightarrow\mathbb{R}^2$, we need to consider how this transformation affects the area of a small rectangle. We can some linear algebra, along with the derivative matrix of $T$, and see that area is scaled by approximately $\text{det}(D\vec{T}(u,v))$.

PICTURE

Below, we repeat the entire statement for change of variables in double integrals.

\begin{proposition}
Let $\vec{T}:\mathbb{R}^2\rightarrow\mathbb{R}^2$ be a $C^1$ function which maps a region $D^*\subset\mathbb{R}^2$ onto a region $D\subset\mathbb{R}^2$, so that $\vec{T}$ restricted to $D^*$ is one-to-one. Suppose $f:D\rightarrow\mathbb{R}$ is an integrable function. Then
\[
\iint_D f(x,y)\;dxdy = \iint_{D^*} f(\vec{T}(u,v))\left|\text{det}(D\vec{T}(u,v))\right|\;dudv.
\]
\end{proposition} 

In this section, we'll see how we can change coordinates in triple integrals.

\section*{Change of variables in triple integrals}

Suppose we have a transformation $\vec{T}:\mathbb{R}^3\rightarrow\mathbb{R}^3$ giving our desired change of coordinates. As with double integrals, we need to consider how this transformation scales. That is, if we have a parallelepiped in $\mathbb{R}^3$, how does applying $T$ affect the volume?

PICTURE

We can approximate the transformation $\vec{T}$ using its derivative matrix, and the absolute value of the determinant of $D\vec{T}$ then gives the scaling factor. So, we have the following result for change of variables in triple integrals.

\begin{proposition}
Let $\vec{T}:\mathbb{R}^3\rightarrow\mathbb{R}^3$ be a $\mathcal{C}^1$ function which maps a region $D^*\subset\mathbb{R}^3$ onto a region $D\subset\mathbb{R}^3$, so that $\vec{T}$ restricted to $D^*$ is one-to-one. Suppose $f:D\rightarrow\mathbb{R}$ is an integrable function. Then
\[
\iiint_D f(x,y,z)\;dxdydz = \iiint_{D^*} f(\vec{T}(u,v,w))\left|\text{det}(D\vec{T}(u,v,w))\right|\;dudvdw.
\]
\end{proposition}

We'll look at examples of this, where we convert to cylindrical and spherical coordinates.

\section*{Cylindrical coordinates}

\begin{example}
Consider the cylinder $D$ pictured below, bounded by $x^2+y^2=1$, $z=0$, and $z=2$.

PICTURE

Consider the integral $\iiint_D x\;dxdydz$. We'll evaluate this integral by converting to cylindrical coordinates.

The cylinder $D$ can be described in cylindrical coordinates by
\begin{align*}
0\leq &\theta \leq 2\pi\\
0\leq &r\leq 1\\
0\leq z\leq 2,
\end{align*}
and these with provide the bounds for our transformed integral.

Recall the relationship between Cartesian coordinates and cylindrical coordinates,
\begin{align*}
x &= r\cos \theta\\
y &= r\sin\theta\\
z &= z.
\end{align*}
Then if $\vec{T}(\theta, r, z) = (r\cos\theta, r\sin\theta, z)$ is the transformation taking cylindrical coordinate to Cartesian coordinates, we have
\begin{align*}
\text{det}(D\vec{T}) &= \text{det}\begin{pmatrix}
\frac{\partial}{\partial \theta} r\cos\theta & \frac{\partial}{\partial r} r\cos\theta & \frac{\partial}{\partial z} r\cos\theta\\
\frac{\partial}{\partial \theta} r\sin\theta & \frac{\partial}{\partial r} r\cos\theta & \frac{\partial}{\partial z} r\sin\theta\\
\frac{\partial}{\partial \theta} z & \frac{\partial}{\partial r} r\cos\theta & \frac{\partial}{\partial z} z
\end{pmatrix}\\
&= \text{det}\begin{pmatrix}
-r\sin\theta & \cos\theta & 0\\
r\cos\theta & \sin\theta & 0\\
0 & 0 & 1
\end{pmatrix}\\
&= -r\sin^2\theta -r\cos^2\theta\\
&= -r.
\end{align*}
Since $r$ is nonnegative, the absolute value of this is $r$. Now, we can complete our change of variables.
\[
\iiint_D x\;dxdydz = \int_0^2\int_0^1\int_0^{2\pi}(r\cos\theta)r\;d\theta drdz
\]
We can then evaluate the integral.
\begin{align*}
\int_0^2\int_0^1\int_0^{2\pi}(r\cos\theta)r\;d\theta drdz &= \int_0^2\int_0^1(r^2\sin\theta)_0^{2\pi}\; drdz\\
&= \int_0^2\int_0^1 0\; drdz\\
&= 0
\end{align*}
\end{example}

In general, the change in differential from cylindrical coordinates to Cartesian coordinates is given by
\[
dxdydz = r\;drd\theta dz.
\]

\section*{Spherical coordinates}

\begin{example}
Consider the solid sphere $D$ of radius $1$ centered at the origin.

PICTURE

The volume of this sphere is given by
\[
\text{volume}(D) = \iiint_D 1\;dV.
\]
We'll compute this volume by changing to spherical coordinates. In spherical coordinates, the sphere $D$ can be described with the inequalities
\begin{align*}
0\leq &\rho\leq 1,\\
0\leq &\theta\leq 2\pi,\\
0\leq &\phi\leq \pi.
\end{align*}
These will provide the bounds for our iterated integral.

Now, consider the transformation
\[
\vec{T}(\rho,\theta,\phi) = (\rho\cos\theta\sin\phi, \rho\sin\theta\sin\phi,\rho\cos\phi),
\]
which converts from spherical coordinates to Cartesian coordinates. We find the scaling factor for this transformation.
\begin{align*}
\text{det}(D\vec{T}) &= \text{det}\begin{pmatrix}
\frac{\partial}{\partial \rho}\rho\cos\theta\sin\phi & \frac{\partial}{\partial \theta}\rho\cos\theta\sin\phi & \frac{\partial}{\partial \phi}\rho\cos\theta\sin\phi\\
\frac{\partial}{\partial \rho} \rho\sin\theta\sin\phi & \frac{\partial}{\partial \theta}\rho\sin\theta\sin\phi & \frac{\partial}{\partial \phi}\rho\sin\theta\sin\phi\\
\frac{\partial}{\partial \rho}\rho\cos\phi & \frac{\partial}{\partial \theta}\rho\cos\phi & \frac{\partial}{\partial \phi}\rho\cos\phi
\end{pmatrix}\\
&= \text{det}\begin{pmatrix}
\cos\theta\sin\phi & -\rho\sin\theta\sin\phi & \rho\cos\theta\cos\phi\\
\sin\theta\sin\phi & \rho\cos\theta\sin\phi & \rho\sin\theta\cos\phi\\
\cos\phi & 0 & -\rho\sin\phi
\end{pmatrix}\\
&= -\cos\phi\left(-\rho^2\sin^2\theta\sin\phi\cos\phi - \rho^2\cos^2\theta\sin\phi\cos\phi\right) + \rho\sin\theta\left(\rho\cos^2\theta\sin^2\phi + \rho\sin^2\theta\sin^2\phi\right)\\
&= -\cos\phi\left(-\rho^2\sin\phi\cos\phi\right) + \rho\sin\theta\left(\rho\sin^2\phi\right)\\
&= \rho^2\sin\phi\cos^2\phi + \rho^2\sin^3\phi\\
&= \rho^2\sin\phi
\end{align*}
Notice that this is nonnegative, since $0\leq \phi\leq \pi$. Now, we can complete our change of variables.
\[
\iiint_D 1\;dV = \int_0^{2\pi}\int_0^\pi\int_0^1 (\rho^2\sin\phi)\;d\rho d\phi d\theta
\]
We then evaluate our integral to find the volume of the sphere.
\begin{align*}
\int_0^{2\pi}\int_0^{\pi}\int_0^1 (\rho^2\sin\phi)\;d\rho d\phi d\theta &= \int_0^{2\pi}\int_0^{\pi}\left(\frac{1}{3}\rho^3\sin\phi\right)_0^1 \; d\phi d\theta\\
&= \int_0^{2\pi}\int_0^{\pi}\left(\frac{1}{3}\sin\phi\right)\; d\phi d\theta\\
&= \int_0^{2\pi}\left(-\frac{1}{3}\cos\phi\right)_0^{\pi}\; d\theta\\
&= \int_0^{2\pi}\left(\frac{2}{3}\right)\; d\theta\\
&= \frac{4}{3}\pi
\end{align*}
Thus, we have that the volume of the unit sphere is $\frac{4}{3}\pi$.
\end{example}

In general, the change in differential from spherical to Cartesian coordinates is
\[
dxdydz = \rho^2\sin\phi\;d\rho d\theta d\phi.
\]

\end{document}