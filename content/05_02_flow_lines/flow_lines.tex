\documentclass{ximera}

\graphicspath{{./graphics/}}

\title{Flow Lines}
\begin{document}
\begin{abstract}
\end{abstract}
\maketitle

In this activity, we define the flow lines of a vector field and how to find them.

\section*{Definition of Flow Lines}

Imagine you have a vector field representing gravitational force in space, and that you have a spaceship floating around in space. The spaceship will move in the direction of the gravitational force, following the vector at its position. As the space ship continues to float through space, it will continue to move in the direction prescribed by the vector field, and trace out a path through space.

PICTURE

A path like this is called a flow line of the vector field. This is the path that ``matches'' the vectors as it moves through the vector fields. This means that the vectors in the vector field should be tangent to the path, and they will actually be the tangent vectors to the path. This leads us to the definition of the flow lines of a vector field.

\begin{definition}
Let $\vec{F}:X\subset \mathbb{R}^n\rightarrow\mathbb{R}^n$ be a vector field, and let $\vec{x}:I\subset \mathbb{R}\rightarrow\mathbb{R}^n$ be a path in $\mathbb{R}^n$. Then we say that $\vec{x}$ is a \emph{flow line} of $\vec{F}$ if 
\[
\vec{x}'(t) = \vec{F}(\vec{x}(t))
\]
for all $t\in I$.
\end{definition}

EXAMPLES 


\section*{How to Find Flow Lines}

STUFF


\end{document}