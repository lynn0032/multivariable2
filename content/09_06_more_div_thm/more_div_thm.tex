\documentclass{ximera}  

%\usepackage{ esint }

\title{More on the Divergence Theorem}  
\author{Melissa Lynn}
\outcome{Understand how the divergence theorem can be used.}

\begin{document}  
\begin{abstract}  
\end{abstract}  
\maketitle 

We've seen that the divergence theorem related a flux integral over a the boundary of a region in $\mathbb{R}^3$ with a triple integral of the divergence over the region.

\begin{theorem}
\textbf{Divergence Theorem.} Let $W$ be a solid region in $\mathbb{R}^3$, with boundary $\partial W$. Suppose that $\partial W$ consists of finitely many orientable, piecewise smooth, and closed surfaces, which are positively oriented with respect to $W$. Let $\vec{F}$ be a $\mathcal{C}^1$ vector field defined on $W$. Then
\[
\oiint_{\partial W} \vec{F}\cdot d\vec{S} = \iiint_W \nabla\cdot \vec{F}\;dV.
\]
\end{theorem}

We'll now look at some examples of how the divergence theorem can be applied to simplify computation.

\section*{Divergence theorem examples}

\begin{example}
Let $S$ be the unit sphere in $\mathbb{R}^3$, oriented with the outward pointing normal vector.

PICTURE

Let $\vec{F}(x,y,z) = \left(\cos(ye^z), x^{100}z^{1000}, e^{\sin(xy^2)}\right)$, and we'll consider the flux integral
\[
\iint_S\vec{F}\cdot d\vec{S}.
\]
If we tried to compute this flux integral directly, it would be very difficult to work with the vector field $\vec{F}$. Instead, we'll apply the divergence theorem.

Notice that $S$ is the boundary of the solid unit sphere $D$, and $S$ is positively oriented relative to $D$. So, we can apply the divergence theorem, and we have
\[
\oiint_{S} \vec{F}\cdot d\vec{S} = \iiint_D \nabla\cdot \vec{F}\;dV.
\]
Before we begin to compute the triple integral, we need to find the divergence of $\vec{F}$.
\begin{align*}
\nabla\cdot \vec{F}(x,y,z) &= \nabla \cdot \left(\cos(ye^z), x^{100}z^{1000}, e^{\sin(xy^2)}\right)\\
&= \frac{\partial}{\partial x}\cos(ye^z) + \frac{\partial}{\partial y}x^{100}z^{1000} +  \frac{\partial}{\partial z}e^{\sin(xy^2)}\\
&= 0
\end{align*}
Since the divergence of $\vec{F}$ is zero, we have
\[
 \iiint_D \nabla\cdot \vec{F}\;dV = 0,
\]
so $\oiint_{S} \vec{F}\cdot d\vec{S}  = 0$ as well. 
\end{example}

\begin{example}
Consider the solid unit cube $D = [0,1]\times [0,1]\times [0,1]$ in $\mathbb{R}^3$,  and let $\vec{F}(x,y,z) = \left(x^2 + \sin(yz), e^{z}, xz\right)$. Consider the flux integral
\[
\oiint_{\partial D} \vec{F}\cdot d\vec{S},
\]
where $\partial D$ is oriented with the outward pointing normal vector.

Evaluating this flux integral directly would require splitting it up into six integrals, one for each face of the unit cube. Instead, it will be much easier to use the divergence theorem. From this, we have
\[
\oiint_{\partial D} \vec{F}\cdot d\vec{S} = \iiint_D \nabla\cdot \vec{F}\;dV.
\]
We'll start by finding the divergence of $\vec{F}$.
\begin{align*}
\nabla\cdot \vec{F}(x,y,z) &= \nabla\cdot \left(x^2 + \sin(yz), e^{z}, xz\right)\\
&=\frac{\partial}{\partial x}(x^2 + \sin(yz)) + \frac{\partial}{\partial y}(e^{z}) + \frac{\partial}{\partial z}(xz)\\
&= 2x+0+x\\
&= 3x
\end{align*}
Now, we'll integrate the divergence over the solid unit cube.
\begin{align*}
 \iiint_D \nabla\cdot \vec{F}\;dV &= \int_0^1\int_0^1\int_0^1 3x\;dzdydx\\
 &= \int_0^1\int_0^13x\;dydx\\
 &= \int_0^1 3x\;dx\\
 &= \left(\frac{3}{2}x^2\right)_{x=0}^{x=1}\\
 &= \frac{3}{2}
\end{align*}
So, using the divergence theorem, we have
\[
\oiint_{\partial D} \vec{F}\cdot d\vec{S} = \frac{3}{2}.
\]
\end{example}

\begin{example}
Let $S$ be the upper half of the unit sphere, oriented with the downward pointing normal vector.

PICTURE

Let $\vec{F} = \left(y\sin(z), xz^10, z+1\right)$, and consider the surface integral $\iint_S\vec{F}\cdot d\vec{S}$. This isn't the simplest vector field, so we would like to find an alternative approach. However, $S$ is not a closed surface, so it isn't immediately obvious how we could apply the divergence theorem.

If we let $W$ be the solid upper hemisphere, then the boundary of $W$ is $S\cup D$, where $D$ is the unit disc in the $xy$-plane.

PICTURE

We do still have an issue here - $S$ was oriented with the \emph{downward} pointing normal vector, so is not positively oriented relative to $W$. Fortunately, we can account for this by writing $\partial W = -S\cup D$, where $D$ is oriented with the downward pointing normal vector.

PICTURE

Now, we can apply the divergence theorem over the region $W$ and it's boundary. This gives us
\begin{align*}
\oiint_{-S\cup D} \vec{F}\cdot d\vec{S} = \iiint_W \nabla\cdot \vec{F}\;dV.
\end{align*}
Let's start with the right side of this equation, and we'll compute the divergence of $\vec{F}$.
\begin{align*}
\nabla\cdot \vec{F} &= \nabla \cdot \left(y\sin(z), xz^10, z+1\right)\\
&= \frac{\partial}{\partial x}y\sin(z) + \frac{\partial}{\partial y}xz^10 \frac{\partial}{\partial z} (z+1)\\
&= 0+0+1\\
&= 1
\end{align*}
When we integrate this over $W$, we have
\begin{align*}
\iiint_W 1\;dV &= (\text{volume of }W)\\
&= \frac{2}{3}\pi,
\end{align*}
since $W$ is half of the solid unit sphere.

Next, let's look at the left side of the equation, $\oiint_{-S\cup D} \vec{F}\cdot d\vec{S}$. Splitting this up into two surface integrals, we have
\[
\oiint_{-S\cup D} \vec{F}\cdot d\vec{S} = -\iint_S\vec{F}\cdot d\vec{S} + \iint_D\vec{F}\cdot d\vec{S}.
\]
We'll now compute the surface integral $\iint_D\vec{F}\cdot d\vec{S}$ directly. At first glance, this might not seem like much of an improvement over our original problem. However, here we are integrating over the unit disc in the $xy$-plane, where $z=0$. When $z=0$, the vector field $\vec{F}$ becomes much simpler. This will make the computation reasonable.

We can parametrize $D$ as $\vec{X}(s,t) = (s\cos t, s\sin t, 0)$ for $0\leq s\leq 1$ and $0\leq t\leq 2\pi$. To compute the surface integral, we compute the normal vector, $X_s\times X_t$.
\begin{align*}
X_s(s,t) \times X_t(s,t) &= (\cos t, \sin t, 0)\times (-s\sin t, s\cos t, 0)\\
&= \text{det}\begin{pmatrix}
\vec{i} & \vec{j} & \vec{k}\\
\cos t & \sin t & 0\\
-s\sin t & s\cos t & 0
\end{pmatrix}\\
&= (s\cos^2 t + s\sin^2 t)\vec{k}
&= (0,0,s)
\end{align*}
This is the upward pointing normal vector, so we need to adjust with a sign change. We now compute the surface integral.
\begin{align*}
\iint_D \vec{F}\cdot d\vec{S} &= \int_0^1\int_0^{2\pi}\vec{F}(s\cos t, s\sin t, 0)\cdot (0,0,-s)\;dtds\\
&= \int_0^1\int_0^{2\pi} (0,0,1)\cdot (0,0,-s)\;dtds\\
&= \int_0^1\int_0^{2\pi} -s\;dtds\\
&= \int_0^1 -2\pi s\;ds\\
&= \left(-\pi s^2\right)_{s=0}^{s=1}\\
&= -\pi
\end{align*}
Now we have enough information to find $\iint_S\vec{F}\cdot d\vec{S}$. Putting all of this together, we have
\begin{align*}
\iint_S\vec{F}\cdot d\vec{S} &= \iint_D\vec{F}\cdot d\vec{S} - \iiint_W \nabla\cdot \vec{F}\;dV\\
 &= -\pi -\frac{2}{3}\pi\\
 &= -\frac{5}{3}\pi,
\end{align*}
Completing our computation.

In this example, we were able to evaluate a surface integral, by creating a closed surface to which we could apply the divergence theorem.
\end{example}

\section*{Geometric interpretation of divergence}

We'll now use the divergence theorem for an extra check on our geometric understanding of divergence, as local expansion or contraction of a vector field.

Let $B_r$ be a small, solid ball of radius $r$ centered at a point $\vec{a}$ in $\mathbb{R}^3$. Let $\vec{F}$ be a $\mathcal{C}^1$ vector field defined near $\vec{a}$. By the divergence theorem, we have
\[
\oiint_{\partial B_r} \vec{F}\cdot d\vec{S} = \iiint_{B_r} \nabla\cdot \vec{F}\;dV
\]
Since $B_r$ is a small ball, we can approximate the triple integral $\iiint_{B_r} \nabla\cdot \vec{F}\;dV$ with
\[
(\text{volume of }B_r)\left(\nabla\cdot \vec{F}\right) = \pi r \left(\nabla\cdot \vec{F}\right) .
\]
So, we have
\[
\oiint_{\partial B_r} \vec{F}\cdot d\vec{S} \approx \pi r \left(\nabla\cdot \vec{F}\right),
\]
which gives us 
\[
\nabla\cdot \vec{F} \approx \frac{1}{\pi r}\oiint_{\partial B_r} \vec{F}\cdot d\vec{S}.
\]
When we take $r\rightarrow 0$, this approximation because more and more accurate, so we have
\[
\nabla\cdot \vec{F} = \lim_{r\rightarrow 0} \frac{1}{\pi r}\oiint_{\partial B_r} \vec{F}\cdot d\vec{S}.
\]
The flux integral on the right measures the flow across the small sphere centered at $\vec{a}$, so by taking the limit, we are measuring how much the vector field is flowing ``into'' or ``out of'' the point $\vec{a}$, giving us local contraction or expansion.

Thus, we see that the divergence of a vector field measures the local expansion or contraction of a vector field.


\end{document}