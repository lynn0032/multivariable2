\documentclass{ximera}  

\title{Change of Variables in Double Integrals}  
\author{Melissa Lynn}
\outcome{Understand how to change variables.}

\begin{document}  
\begin{abstract}  
\end{abstract}  
\maketitle  

When performing a $u$-substitution on a definite integral in single variable calculus, we paid careful attention to the following steps.
\begin{itemize}
\item Changing the variable
\item Changing the differential
\item Changing the interval of integration
\end{itemize}
For example, making a $u$-substitution in the integral $\int_1^3 2x e^{x^2}\;dx$, we chose the change of variable $u=x^2$, changed the differential to $du = g'(x)dx$, and changed the bounds of integration to $1^2 = 1$ and $3^2 = 9$. This gave us
\[
\int_1^3 2x e^{x^2}\;dx = \int_1^9 e^u\;du,
\]
which we could then evaluate.

We'll now turn our attention to change of variables in double integrals, which will be a useful tool for evaluating these integrals. When we used $u$-substitution in single variable calculus, we often focused on making the \emph{integrand} easier to integrate. That is, we wished to transform the integrand, so that it had a more obvious antiderivative.

In contract, when performing a change of variables on double integrals, we're often more focused on how the change of variables affects the region of integration. Consider the following regions in $\mathbb{R}^2$. The region on the left can most easily be described using polar coordinates, while the region on the right can most easily be described using a linear change of coordinates.

PICTURE

For double integrals, rather than finding complicated combinations of elementary regions to describe a domain, it will be useful to make a change of coordinates in order to describe the domain of integration more simply.

\section*{A linear change of coordinates}

Let's look at what happens when we make a linear change of coordinates. Consider the region $D$ below, and consider the double integral $\iint_D (x+y)\;dxdy$.

PICTURE

The region $D$ can most easily be described with the inequalities
\begin{align*}
1\leq 2x+y \leq 3,\\
0\leq x-3y \leq 1.
\end{align*}

This motivates the change of coordinates $u=2x+y$ and $v=x-3y$. This is a linear change of coordinates, which can be represented with a matrix. That is, we can write
\[
\begin{pmatrix}
u\\ v
\end{pmatrix}
 = \begin{pmatrix}
2 & 1\\
1 & -3
\end{pmatrix}
\begin{pmatrix}
x\\ y
\end{pmatrix}.
\]
Now, let's look at how this linear transformation changes the region of integration. In $(u,v)$-coordinates, our region of integration can be described as
\begin{align*}
1\leq u \leq 3,\\
0\leq v \leq 1.
\end{align*}
This corresponds to a parallelogram in $(x,y)$-coordinates.

PICTURE

We can see that the linear transformation changes the domain of integration, and we will need to account for this as we make our change of coordinates.

We'll begin with the simplest step, which is changing our integrand. If we take the equations $u=2x+y$ and $v=x-3y$, and solve for $x$ and $y$, we have
\begin{align*}
x &= \frac{3}{7}u + \frac{1}{7}v,
y &= \frac{1}{7}u - \frac{2}{7}v.
\end{align*}
So, our integrand becomes
\begin{align*}
x+y &= \left(\frac{3}{7}u + \frac{1}{7}v\right)+\left(\frac{1}{7}u - \frac{2}{7}v\right)\\
&= \frac{4}{7}u - \frac{1}{7}v.
\end{align*}

Next, let's look at our bounds of integration. Since the region of integration can be described using the bounds
\begin{align*}
1\leq u \leq 3,\\
0\leq v \leq 1,
\end{align*}
our integral will have the form
\[
\int_0^1\int_1^3 ?\;dudv.
\]

Finally, we need to consider how the change of coordinates affects the differential. That is, what will the area expansion factor be?

When we apply the linear transformation $\begin{pmatrix}
2 & 1\\
1 & -3
\end{pmatrix}$ to any rectangle, the area of the rectangle is multiplied by $\left|\text{det}\begin{pmatrix}
2 & 1\\
1 & -3
\end{pmatrix}\right| = \left|2\cdot -3 - 1\cdot 1\right| = 7$.

So, when we change to $(u,v)$-coordinates, we need to divide by $7$, in order to account for this change in area. This gives us the change of coordinates,
\[
\iint_D (x+y)\;dxdy =\int_0^1\int_1^3 \left(\frac{4}{7}u - \frac{1}{7}v\right)\frac{1}{7}dudv.
\]
From here, we can evaluate the integral.
\begin{align*}
\int_0^1\int_1^3 \left(\frac{4}{7}u - \frac{1}{7}v\right)\frac{1}{7}dudv
&= \int_0^1 \left(\frac{2}{49}u^2 - \frac{1}{49}vu\right)|_{u=1}^{u=3}dv\\
&= \int_0^1 \left(\frac{2}{49}3^2 - \frac{1}{49}v\cdot 3\right) - \left(\frac{2}{49}1^2 - \frac{1}{49}v\cdot 1\right)dv\\
&= \int_0^1 \left(\frac{16}{49} - \frac{2}{49}v\right)dv\\
&= \left(\frac{16}{49}v - \frac{1}{49}v^2\right)|_{v=0}^{v=1}\\
&= \frac{15}{49}
\end{align*}

\section*{Change of variables in general}

Now, let's look at change of variables for double integrals in general, where the change of coordinates might not be linear.

Suppose we are computing a double integral $\iint_D f(x,y)dxdy$, and suppose that $\vec{T}:\mathbb{R}^2\rightarrow\mathbb{R}^2$ is a change of coordinates which maps some region $D^*$ onto $D$. Suppose further that $T$ is $\mathcal{C}^1$, and that $T$ is one-to-one on $D^*$.

PICTURE

We think of the domain of $T$ as being in $(u,v)$-coordinates, and the codomain of $T$ as being in $(x,y)$-coordinates. So, the region $D^*$ would be described in terms of $u$ and $v$, and is mapped to $D$, which is described in terms of $x$ and $y$.

We can see that our change of coordinates will have the form
\[
\iint_D f(x,y)dxdy = \iint_{D^*} ? dudv,
\]
but we still need to determine how the differential will change, by finding the area expansion factor.

Suppose we have a tiny rectangle $R$ in $D^*$, and we apply the transformation $\vec{T}$ to this rectangle, to obtain a region $\vec{T}(R)$.

PICTURE

We can approximate the transformation $\vec{T}$ with the linear transformation given by its derivative matrix $D\vec{T}$, so the area of $\vec{T}(R)$ can be approximated as
\[
\text{area } \vec{T}(R) \approx \left|\text{det}(D\vec{T})\right|(\text{area }R).
\]
This idea gives us our area expansion factor, and we change the differential according to
\[
dxdy = \left|\text{det}(D\vec{T})\right|\;dudv.
\]
Now, we can fully describe how to change variables in double integrals.

\begin{proposition}
Let $\vec{T}:\mathbb{R}^2\rightarrow\mathbb{R}^2$ be a $C^1$ function which maps a region $D^*\subset\mathbb{R}^2$ onto a region $D\subset\mathbb{R}^2$, so that $\vec{T}$ restricted to $D^*$ is one-to-one. Suppose $f:D\rightarrow\mathbb{R}$ is an integrable function. Then
\[
\iint_D f(x,y)\;dxdy = \iint_{D^*} f(\vec{T}(u,v))\left|\text{det}(D\vec{T}(u,v))\right|\;dudv.
\]
\end{proposition}

\section*{Polar coordinates}

Consider the double integral $\iint_D (x^2+y^2)\;dxdy$, where $D$ is the region below.

PICTURE

The region $D$ can most easily be described using polar coordinates, so that
\begin{align*}
0\leq r\leq 2,\\
\pi/4\leq \theta\leq \pi/2.
\end{align*}
Recall that polar coordinates relate to Cartesian coordinates via
\begin{align*}
x &= r\cos\theta,\\
y &= r\sin\theta.
\end{align*}
Thus, our transformation $T$ is given by $T(r,\theta) = (r\cos\theta, r\sin\theta)$. In order to see how we must change the differential in the double integral, we compute $\left|\text{det}(D\vec{T}(u,v))\right|$.
\begin{align*}
\left|\text{det}(D\vec{T}(u,v))\right| &= \left|\text{det}\begin{pmatrix}\cos\theta & \sin\theta \\ -r\sin\theta & r\cos\theta\end{pmatrix}\right|\\
&= \left|\cos\theta\cdot r\cos\theta - \sin\theta \cdot -r\sin\theta \right|\\
&= \left|r\right|\\
\end{align*}
Since we describe our region using only nonnegative values for $r$, the absolute value is unnecessary. Thus, our differential will change according to
\[
dxdy = r\;drd\theta.
\]
Finally, let's look at our integrand. Using the substitutions $x = r\cos\theta$ and $y = r\sin\theta$, we have
\[
x^2 +y^2 = r^2\cos^2\theta + r^2\sin^2\theta = r^2.
\]
Putting all of this together, we have our change of variables for the double integral, and we can evaluate our transformed integral.
\begin{align*}
\iint_D (x^2+y^2)\;dxdy &= \int_{\pi/4}^{\pi/2}\int_0^2 r^2\cdot r\;drd\theta\\
&= \int_{\pi/4}^{\pi/2}\int_0^2 r^3\;drd\theta\\
&= \int_{\pi/4}^{\pi/2}\frac{1}{4}r^4|_{r = 0}^{r = 2}d\theta\\
&= \int_{\pi/4}^{\pi/2}4d\theta\\
&= 4\left(\frac{\pi}{2}-\frac{\pi}{4}\right)\\
&= \pi.
\end{align*}

Notice that any time we change to polar coordinates, we will complete the same computation for the area expansion factor. That is, for changing to polar coordinates, we always have
\[
dxdy = r\;drd\theta.
\]


\end{document}